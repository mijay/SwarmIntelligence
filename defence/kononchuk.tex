\documentclass[14pt]{beamer}
\usepackage[T2A]{fontenc}
\usepackage[utf8]{inputenc}
\usepackage[russian,english]{babel}
\usepackage{tabularx}
\usetheme{Warsaw}


\title{Система поддержки алгоритмов коллективного разума}
\author[Конончук Д.О.]{Дмитрий Конончук}
\institute{
    Уральский Государственный Университет им. А.М.Горького \vspace{0.9em}

    Научный руководитель: Окуловский Юрий Сергеевич \vspace{0.9em}
}
\date{\footnotesize 2010}
\begin{document}

\begin{frame}
    \titlepage
\end{frame}

\begin{frame}[c]
    \frametitle{Типичные аналогии}
    \begin{columns}
        \begin{column}{60mm}
            \center{
                \includegraphics[width=40mm]{1.png}
                
                Муравейник
            }
        \end{column}
        \begin{column}{60mm}
            \center{
                \includegraphics[width=40mm]{2.png}

                Улей
            }
        \end{column}
    \end{columns}
\end{frame}

\begin{frame}
    \frametitle{Первое упоминание}
    \LARGE
    \begin{center}
        1989 \vspace{1cm}

        Gerardo Beni, Jing Wang \vspace{1cm}

        Swarm Intelligence in Cellular Robotic Systems
    \end{center}
\end{frame}

\begin{frame}
    \frametitle{Преимущества фреймворков}
    \begin{itemize}
        \item Простота реализации приложений.
        \item Упрощение повторного использования кода.
        \item Упрощение сравнения приложений.
        \item Лучшая реализация сервисных функций.
        \item Структурированный подход к предметной области.
        \item Однотипный подход для решения многих задач.
        \item Упрощение обучения.
    \end{itemize}
\end{frame}

\begin{frame}
    \frametitle{Основные свойства АКР}
    \begin{itemize}
        \item Алгоритм работает в некотором адресуемом пространстве.
        \item Единственная активная сущность --- агент.
        \item Агент может перемещаться по пространству и изменять данные, ассоциированные с его ячейками.
        \item Все действия агентов на каждой итерации независимы друг от друга.
    \end{itemize}
\end{frame}

\begin{frame}[<+->]
    \frametitle{Требования к системе}
    \begin{enumerate}
        \item Единая и удобная программная модель.
        \item Высокая производительность.
        \item Распараллеливание и распределение.
        \item Визуализация.
    \end{enumerate}
\end{frame}

\begin{frame}
    \frametitle{Задачи}
    \setbeamercovered{transparent=50}
    \begin{enumerate}
        \item<1> Разработка единой архитектуры.
        \item<1> Улучшение ее юзабилити.
        \item<1> Ее оптимизация под распараллеливание и транзакции.
        \item<1-2> Реализация транзакций.
        \item<1> Реализация распараллеливания и распределения.
        \item<1> Реализация визуализации.
        \item<1>  $\cdots$
    \end{enumerate}
\end{frame}

%\begin{frame}
%    \frametitle{Технологии}
%    \begin{description}
%        \item[Архитектура:] SOLID
%        \item[Платформа:] C\# .NET 4.0
%        \item[Контракты:] CodeContracts
%        \item[Распараллеливание:] PLINQ, TPL, LINQ to Events
%        \item[Распределение:] WCF
%    \end{description}
%\end{frame}

\begin{frame}
    \begin{center}
        Спасибо за внимание
    \end{center}
\end{frame}
\end{document}
