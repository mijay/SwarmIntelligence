\documentclass[14pt]{beamer}
\usepackage[T2A]{fontenc}
\usepackage[utf8]{inputenc}
\usepackage[russian,english]{babel}
\usepackage{tabularx}
\usetheme{Warsaw}


\title{Система поддержки алгоритмов коллективного разума}
\author[Конончук Д.О.]{Дмитрий Конончук}
\institute{
    Уральский Федеральный Университет \vspace{0.9em}

    Научный руководитель: Окуловский Юрий Сергеевич \vspace{0.9em}
}
\date{\footnotesize 2011}
\begin{document}

\begin{frame}
    \titlepage
\end{frame}

\begin{frame}
    \frametitle{Принцип АКР}
    Множество простых агентов в совокупности решают сложную задачу.

    \vspace{2ex}
    Аналогии в естественном мире: муравейник, улей\dots
\end{frame}

\begin{frame}
    \frametitle{Основополагающие работы}
    \begin{center}
        {\fontsize{16pt}{16pt}Ant Colony Optimization}

        {\small
            1996 {\it Dorigo M., Maniezzo V., Colorni A.} The Ant System: Optimization by a colony of cooperating agents.
        }

        \vspace{6ex}
        {\fontsize{16pt}{16pt}Particle Swarm Optimization}

        {\small
            2001 {\it Kennedy J., Eberhart R. C.} Swarm intelligence.
        }
    \end{center}
\end{frame}

\begin{frame}
    \frametitle{Основные свойства АКР}
    \begin{itemize}
        \item Алгоритм итеративный.
        \item Оперирует агентами, расположенными в некотором дискретном пространстве.
        \item Агенты атомарны и децентрализованы.
        \item С точками пространства ассоциированы данные.
        \item Агент может перемещаться по пространству, читать и модифицировать данные.
        \item Все действия агентов на каждой итерации независимы друг от друга.
    \end{itemize}
\end{frame}

\begin{frame}
    \frametitle{Реализация АКР}
    {\bf Хочется реализовать: } логику агентов.

    \vspace{2ex}
    \pause
    {\bf Приходится реализовывать: } пространство, размещение данных на нем, механику перемещений агентов,~\dots
\end{frame}

\begin{frame}
    \frametitle{Задачи фреймворка}
    \begin{enumerate}
        \item Единая компонентная программная модель.
        \item Размещение агентов и данных в пространстве.
        \item Механика перемещений и изменений данных.
        \item Логгирование и статистика.
        \item Визуализация.
        \item Реализация часто используемых компонентов.
    \end{enumerate}
\end{frame}

\begin{frame}
    \frametitle{Проблемы}
    \begin{itemize}[<+->]
        \item Нетривиальные пространства:
        \begin{itemize}[<.->]
            \item большие,
            \item бесконечные,
            \item сложно вычислимые,
            \item динамические.
        \end{itemize}
        \item Изоляция действий агентов.
        \item Распараллеливание вычислений.
        \item Визуализация абстрактных пространств.
    \end{itemize}
\end{frame}

\begin{frame}
    \frametitle{Реализация}
    \begin{description}
        \item[Платформа:] C\# .NET 4.0
        \item[Визуализация:] WPF
        \item[Распараллеливание:] TPL, PLINQ
        \item[Контракты:] CodeContracts
    \end{description}
\end{frame}

\begin{frame}
    \frametitle{Преимущества фреймворка}
    \begin{itemize}
        \item Простота создания приложений.
        \item Лучшая реализация сервисных функций.
        \item Структурированный подход к предметной области.
        \item Упрощение обучения.
        %\item Упрощение повторного использования кода.
        \item Возможность сравнения приложений.
    \end{itemize}
\end{frame}

%\begin{frame}
%    \frametitle{Технологии}
%    \begin{description}
%        \item[Архитектура:] SOLID
%        \item[Платформа:] C\# .NET 4.0
%        \item[Контракты:] CodeContracts
%        \item[Распараллеливание:] PLINQ, TPL, LINQ to Events
%        \item[Распределение:] WCF
%    \end{description}
%\end{frame}

\begin{frame}
    \begin{center}
        Спасибо за внимание
    \end{center}
\end{frame}

%\begin{frame}
%    \frametitle{АКР vs МАА}
%    \begin{columns}[t]
%        \begin{column}{60mm}
%            \center{
%                {\LARGE АКР}
%
%                \vspace{2ex}
%                \fontsize{10pt}{10pt}
%                \begin{itemize}[<+->]
%                    \item Множество децентрализованых агентов.
%                    \item Агенты расположены в пространстве.
%                    \item Агенты взаимодействуют через данные.
%                    \item Агенты принимают решения независимо.
%                    \item Вычисления через движение.
%                \end{itemize}
%            }
%        \end{column}
%        \begin{column}{60mm}
%            \center{
%                {\LARGE МАА}
%
%                \vspace{2ex}
%                \fontsize{10pt}{10pt}
%                \setcounter{beamerpauses}{1}
%                \begin{itemize}[<+->]
%                    \item Множество децентрализованых агентов.
%                    \item Агенты абстрактны.\newline
%                    \item Агенты взаимодействуют через сообщения.
%                    \item Агенты полностью независимы.\newline
%                    \item Вычисления через коммуникацию.
%                \end{itemize}
%            }
%        \end{column}
%    \end{columns}
%\end{frame}

%\begin{frame}
%    \frametitle{АКР vs МАА}
%    \begin{columns}[t]
%        \begin{column}{60mm}
%            \center{
%                {\LARGE АКР}
%
%                \vspace{2ex}
%                \fontsize{10pt}{10pt}
%                Задачи поиска и оптимизации.
%            }
%        \end{column}
%        \begin{column}{60mm}
%            \center{
%                {\LARGE МАА}
%
%                \vspace{2ex}
%                \fontsize{10pt}{10pt}
%                Задачи распределения ресурсов и составления расписаний.
%            }
%        \end{column}
%    \end{columns}
%\end{frame}

\end{document}
