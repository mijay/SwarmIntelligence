\documentclass[14pt]{beamer}
\usepackage[T2A]{fontenc}
\usepackage[utf8]{inputenc}
\usepackage[russian,english]{babel}
\usepackage{tabularx}
\usepackage{pscyr}
\usetheme{Warsaw}


\title{Система поддержки алгоритмов коллективного разума}
\author[Конончук Д.О.]{Дмитрий Конончук}
\institute{
    Уральский Федеральный Университет\\имени Первого Президента России\\Б.Н.Ельцина\vspace{0.5cm}

    Научный руководитель: Окуловский Юрий Сергеевич \vspace{0.9em}
}
\date{\footnotesize 2012}
\begin{document}

\begin{frame}
    \titlepage
\end{frame}

\begin{frame}
    \frametitle{Определение}
    
    \begin{description}
        \item[АКР] --- алгоритмы исскуственного интеллекта, основаные на принципе решения задачи множеством простых агентов.
    \end{description}
\end{frame}

\begin{frame}
    \frametitle{Первая работа}
    \LARGE
    \begin{center}
        1991 \vspace{0.5cm}

        A. Colorni, M. Dorigo et V. Maniezzo \vspace{0.5cm}

        Distributed Optimization by Ant Colonies
    \end{center}
\end{frame}

\begin{frame}
    \frametitle{Система поддержки}
    \begin{itemize}
        \item Фиксирует структуру алгоритма.
        \item Предоставляет средства для упрощения его выражения.
        \item Реализует сервисные функции.
    \end{itemize}
\end{frame}

% \begin{frame}
    % \frametitle{Фреймворки}
    % \begin{columns}[t]
        % \begin{column}{0.4\textwidth}
            % Нейронные сети
            % \begin{itemize}
                % \item SNNS
                % \item Emergnet
                % \item JavaNNS
                % \item NeuralLab
                % \item Neuron
                % \item Gans
            % \end{itemize}
        % \end{column}
        % \begin{column}{0.6\textwidth}
            % Генетические алгоритмы
            % \begin{itemize}
                % \item EvoJ
                % \item CuberGA
                % \item GAlib
                % \item JAGA
                % \item GAUL
                % \item Gans
            % \end{itemize}
        % \end{column}
    % \end{columns}
% \end{frame}

\begin{frame}
    \frametitle{Возможные фреймворки для АКР}
        Мульти-агентное моделирование: JADE, MASS, Repast, breve \ldots
        \vspace{1cm}        
        
        Алгоритмы эволюционного моделирования:  EO Evolutionary Computation Framework, ECF \ldots
\end{frame}


\begin{frame}
    \frametitle{Основные свойства АКР}
    \begin{itemize}
        \item Алгоритм работает в некотором адресуемом пространстве.
        \item Единственная активная сущность --- агент.
        \item Агент может перемещаться по пространству и изменять данные, ассоциированные с его ячейками.
        \item Все действия агентов на каждой итерации независимы друг от друга.
    \end{itemize}
\end{frame}

\begin{frame}
    \frametitle{Задачи}
    \setbeamercovered{transparent=50}
    \begin{enumerate}
        \item<1> Разработка единой архитектуры.
        \item<1> Улучшение ее юзабилити.
        \item<1> Ее оптимизация под распараллеливание и транзакции.
        \item<1-2> Реализация транзакций.
        \item<1> Реализация распараллеливания и распределения.
        \item<1> Реализация визуализации.
        \item<1>  $\cdots$
    \end{enumerate}
\end{frame}

\begin{frame}
    \begin{center}
        Спасибо за внимание
    \end{center}
\end{frame}
\end{document}
