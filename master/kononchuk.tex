\documentclass[14pt]{beamer}
\usepackage[T2A]{fontenc}
\usepackage[utf8]{inputenc}
\usepackage[russian,english]{babel}
\usepackage{tabularx}
\usepackage{pscyr}
\usetheme{Warsaw}


\title{Система поддержки алгоритмов коллективного разума}
\author[Конончук Д.О.]{Дмитрий Конончук}
\institute{
    Уральский Федеральный Университет\\имени Первого Президента России\\Б.Н.Ельцина\vspace{0.5cm}

    Научный руководитель: Окуловский Юрий Сергеевич \vspace{0.9em}
}
\date{\footnotesize 2012}
\begin{document}

\begin{frame}
    \titlepage
\end{frame}

\begin{frame}
    \frametitle{Определение}
    
    \begin{description}
        \item[АКР] --- алгоритмы искусственного интеллекта, основанные на принципе решения задачи множеством простых агентов.
    \end{description}
\end{frame}

\begin{frame}
    \frametitle{Первая работа}
    \LARGE
    \begin{center}
        1991 \vspace{0.5cm}

        A. Colorni, M. Dorigo et V. Maniezzo \vspace{0.5cm}

        Distributed Optimization by Ant Colonies
    \end{center}
\end{frame}

% \begin{frame}
    % \frametitle{Система поддержки}
    % Реализация алгоритмов определенного класса.
    % \begin{itemize}
        % \item Фиксирует архитектуру программного представления алгоритма.
        % \item Предоставляет средства для выражения алгоритма в терминах этой архитектуры.
        % \item Обеспечивает функционирование архитектуры.
    % \end{itemize}
% \end{frame}

\begin{frame}
    \frametitle{Возможные фреймворки для АКР}
        Мульти-агентное моделирование: JADE, MASS, Repast, breve \ldots
        \vspace{1cm}        
        
        Алгоритмы эволюционного моделирования:  EO Evolutionary Computation Framework, ECF \ldots
\end{frame}

\begin{frame}
    \frametitle{Основные свойства АКР}
    \begin{itemize}
        \item Алгоритм работает на карте --- адресуемый граф, потенциально бесконечный.
        \item С вершинами и ребрами графа ассоциированы данные.
        \item В вершинах располагаются агенты.
    \end{itemize}
\end{frame}

\begin{frame}
    \frametitle{Агенты АКР}
    \begin{itemize}
        \item Агент --- единственная активная сущность.
        \item Агент может:
            \begin{itemize}
                \item перемещаться по пространству;
                \item читать данные, ассоциированные с его вершинами и ребрами;
                \item изменять их.
            \end{itemize}
        \item Все действия агентов на каждой итерации независимы друг от друга.
    \end{itemize}
\end{frame}

\begin{frame}
    \frametitle{Функции системы поддержки}
    \begin{itemize}
        \item Архитектура программного представления АКР.
        \item Размещение и перемещение агентов в пространстве.
        \item Связывание данных с пространством.
        \item Реализация распараллеливания и конкурентного доступа к данным.
        \item Система событий.
        \item Библиотека компонентов.
    \end{itemize}
\end{frame}

\begin{frame}
    \begin{center}
        Спасибо за внимание
    \end{center}
\end{frame}
\end{document}

